\documentclass[a4paper, 11pt]{article}

\usepackage[french]{babel}
\usepackage[utf8x]{inputenc}
\usepackage[T1]{fontenc}
\usepackage{placeins}
\usepackage{csquotes}
\usepackage{hyperref}
\usepackage{framed}

\author{Florian Thuin \and Cyril de Vogelaere}
\date{\today}
\title{LTECO2300 --- Bibliographie documentée}

\begin{document}
    \maketitle
    \tableofcontents
    \textsc{Cochinaux} Philippe, L'éthique (Que penser de), Namur, Fidelité,
    2008.

    \begin{framed}
        Ce livre nous a principalement servi de complément au cours afin de
        comprendre la différence entre morale et éthique. Les pages traitant la
        désobéissance civile nous été fortement utiles pour développer notre
        raisonnement sur l’éthique de ces désobéissances, pour comprendre la
        différence entre conscience et sur-moi, pour comprendre le lien entre
        éthique et liberté.
    \end{framed}

    \bsc{Arendt} Hannah, Crise of the Republic: Lying in politics, civil
    disobedience, On violence, thoughts on politics and revolution, San Diego:
    Harcourt, Brace and Jovanovich, 1972.

    \begin{framed}
        Le chapitre sur la désobéissance civile de cette monographie nous a été
        utile afin de comprendre les causes de la désobéissance civile dans le
        monde moderne. En effet, ce livre examine les multiples mouvements
        d’oppositions contemporains à l’auteur, étudiant le comportement de ses
        activistes pour tenter de connaître les motifs et d’expliquer les
        causes de leurs actions. Elle identifiera finalement leurs actions à
        l’émergence du progrès moderne et aux échecs des institutions
        gouvernementales américaines.
    \end{framed}


    \bsc{Arendt} Hannah, Du mensonge à la violence. Essais de politique
    contemporaine, trad. G. Durand, Paris, Calmann-Lévy, 1972 (Crise of the
    Republic, 1972).

    \begin{framed}
        Cette version traduite en français du livre précédent nous a permis de
        mieux comprendre certains passages compliqués de la version anglaise.
    \end{framed}


    \bsc{Chenoweth} Erica et STEPHAN Maria J., Why civil resistance works: The
    Strategic Logic of Nonviolent Conflict,  International Security, volume 33,
    issue 1, pages 7--44, en ligne :
    \url{http://belfercenter.ksg.harvard.edu/files/IS3301_pp007-044_Stephan_Chenoweth.pdf}
    (consulté le 20/11/2015).

    \begin{framed}
        Cet article étudie l'efficacité d'actes de résistances (Violent et
        non-violent) entre des individus ou groupe d'individus et des
        institutions étatiques. \newline

        L'intérêt majeur de cet article n'est pas seulement le fait qu'il soit
        l'un des rare à étudier, chiffres à l’appui,  les actes violents aussi
        bien que les actes non violents mais aussi le fait qu’il fasse une
        analyse critique de ces actes, posant diverses hypothèses et les
        prouvant l’une après l’autre. \newline
        Cet article nous a permis de déterminer que les actes non violents ne
        sont pas moins efficaces que leur brutale contrepartie. Ils le sont
        même plus dans une certaine mesure car la pression posée sur les cibles
        de ces résistances civiles est de longue durée et difficile à réprimer
        tout en maintenant une image publique acceptable. \newline

        L’article prouve également comme nous le disons dans notre présentation
        que la violence n’est en rien une solution, que ce soit au niveau de
        force de l’ordre que des protestants.
    \end{framed}


    \bsc{Yagil} Limore, La France terre de refuge et de désobéissance civile
    (1936--1944). Exemple du sauvetage des Juifs, Tome I, éd. Cerf Histoire,
    2010.

    \begin{framed}
        Cette monographie nous a été très utile pour définir et comprendre ce
        qu’était exactement la désobéissance civile, de comprendre en quoi la
        désobéissance civile se distingue des notions voisines telles que
        l’objection de conscience et la résistance civile. \newline

        Elle nous a également permis de comprendre l’émergence de cette idée
        dans nos sociétés gouvernées par une culture de l’obéissance. De mieux
        comprendre le lien de cette désobéissance avec la notion de sujet et la
        notion de Dieu.
    \end{framed}

    \bsc{Cohens} Carl, Civil disobedience: conscience, tactics, and the law,
    New York and London, Colombia University Press, 1971.

    \begin{framed}
        Ce livre de 1971 parle de manière extensive de la désobéissance civile.
        Présentant tout d'abord ce qu’elle est et n'est pas ainsi que les
        différents types de désobéissance civile et leurs punitions légales.
        \newline

        Ce livre présente également différents arguments pour et contre la
        désobéissance civile dans ses chapitres 5 et 6, chapitres qui ont par
        ailleurs été fort utiles à notre lecture car ils présentent
        extensivement la dualité des opinions sur le sujet. \newline

        Le livre présente également l'articulation de la liberté de parole et
        de la désobéissance civique ainsi que les fameux jugements de Nuremberg
        où des citoyens ayant suivis les lois imposées par le régime nazi ont
        été condamné pour ne pas avoir désobéis à la loi nazie et aidé ceux qui
        sont devenus leurs victimes. \newline

        Ce livre fut donc particulièrement utile pour nos forger une opinion
        critique sur le sujet. \newline
    \end{framed}


    \bsc{Züger} Theresa, Re-thinking civil disobedience, Internet Policy Review,
    volume 2, issue 4, 2013,
    \url{http://policyreview.info/articles/analysis/re-thinking-civil-disobedience}
    (Consulté le 18/11/2015)

    \begin{framed}
        Cet article étudie la récente émergence de nouvelles méthodes modernes
        de désobéissance civile. Basée sur les recherches d'Arendt, le but de
        cet article est non seulement d'en faire une analyse critique mais
        surtout de faire changer les lois actuellement en vigueur car, selon
        l’auteur, celle -ci ne seraient plus adaptées depuis l'émergence de ces
        nouvelles méthodes de désobéissance civile. \newline

        L'article s'attarde aussi légèrement sur la légitimité de ces nouvelles
        méthodes avant de conclure par une nouvelle demande de mettre à jour
        les lois. \newline

        Bien que cette article prennent un point de vue plus juridique, il n’en
        reste pas moins intéressant par le fait qu’il nous a permis de mieux
        comprendre des phénomènes récents tels que l'apparition de Wikileaks ou
        l'émergence du mouvement Anonymous. \newline

        Malheureusement dû à un manque de temps, nous n’avons pas eu l’occasion
        d’en inclure les réflexions dans notre présentation de 10 minutes.
    \end{framed}

\end{document}
